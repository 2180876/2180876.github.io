%%%%%%%%%%%%%%%%%%%%%%%%%%%%%%%%%%%%%%%%%%%%%%%%%%%%%%%%%%%%%%%
%% OXFORD THESIS TEMPLATE

% Use this template to produce a standard thesis that meets the Oxford University requirements for DPhil submission
%
% Originally by Keith A. Gillow (gillow@maths.ox.ac.uk), 1997
% Modified by Sam Evans (sam@samuelevansresearch.org), 2007
% Modified by John McManigle (john@oxfordechoes.com), 2015
% Modified by Ulrik Lyngs (ulrik.lyngs@cs.ox.ac.uk), 2018-, for use with R Markdown
%
% Ulrik Lyngs, 25 Nov 2018: Following John McManigle, broad permissions are granted to use, modify, and distribute this software
% as specified in the MIT License included in this distribution's LICENSE file.
%
% John commented this file extensively, so read through to see how to use the various options.  Remember that in LaTeX,
% any line starting with a % is NOT executed.

%%%%% PAGE LAYOUT
% The most common choices should be below.  You can also do other things, like replace "a4paper" with "letterpaper", etc.

% 'twoside' formats for two-sided binding (ie left and right pages have mirror margins; blank pages inserted where needed):
%\documentclass[a4paper,twoside]{templates/ociamthesis}
% Specifying nothing formats for one-sided binding (ie left margin > right margin; no extra blank pages):
%\documentclass[a4paper]{ociamthesis}
% 'nobind' formats for PDF output (ie equal margins, no extra blank pages):
%\documentclass[a4paper,nobind]{templates/ociamthesis}

% As you can see from the line below, oxforddown uses the a4paper size, 
% and passes in the binding option from the YAML header in index.Rmd:
\documentclass[a4paper, nobind]{templates/ociamthesis}


%%%%% ADDING LATEX PACKAGES
% add hyperref package with options from YAML %
\usepackage[pdfpagelabels]{hyperref}
% handle long urls
\usepackage{xurl}
% change the default coloring of links to something sensible
\usepackage{xcolor}

\definecolor{mylinkcolor}{RGB}{0,0,139}
\definecolor{myurlcolor}{RGB}{0,0,139}
\definecolor{mycitecolor}{RGB}{0,33,71}

\hypersetup{
  hidelinks,
  colorlinks,
  linktocpage=true,
  linkcolor=mylinkcolor,
  urlcolor=myurlcolor,
  citecolor=mycitecolor
}


% add float package to allow manual control of figure positioning %
\usepackage{float}

% enable strikethrough
\usepackage[normalem]{ulem}

% use soul package for correction highlighting
\usepackage{color, soulutf8}
\definecolor{correctioncolor}{HTML}{CCCCFF}
\sethlcolor{correctioncolor}
\newcommand{\ctext}[3][RGB]{%
  \begingroup
  \definecolor{hlcolor}{#1}{#2}\sethlcolor{hlcolor}%
  \hl{#3}%
  \endgroup
}
% stop soul from freaking out when it sees citation commands
\soulregister\ref7
\soulregister\cite7
\soulregister\citet7
\soulregister\autocite7
\soulregister\textcite7
\soulregister\pageref7

%%%%% FIXING / ADDING THINGS THAT'S SPECIAL TO R MARKDOWN'S USE OF LATEX TEMPLATES
% pandoc puts lists in 'tightlist' command when no space between bullet points in Rmd file,
% so we add this command to the template
\providecommand{\tightlist}{%
  \setlength{\itemsep}{0pt}\setlength{\parskip}{0pt}}
 
% allow us to include code blocks in shaded environments

% User-included things with header_includes or in_header will appear here
% kableExtra packages will appear here if you use library(kableExtra)
\usepackage{booktabs}
\usepackage{longtable}
\usepackage{array}
\usepackage{multirow}
\usepackage{wrapfig}
\usepackage{float}
\usepackage{colortbl}
\usepackage{pdflscape}
\usepackage{tabu}
\usepackage{threeparttable}
\usepackage{threeparttablex}
\usepackage[normalem]{ulem}
\usepackage{makecell}
\usepackage{xcolor}


%UL set section header spacing
\usepackage{titlesec}
% 
\titlespacing\subsubsection{0pt}{24pt plus 4pt minus 2pt}{0pt plus 2pt minus 2pt}


%UL set whitespace around verbatim environments
\usepackage{etoolbox}
\makeatletter
\preto{\@verbatim}{\topsep=0pt \partopsep=0pt }
\makeatother


%%%%%%% PAGE HEADERS AND FOOTERS %%%%%%%%%
\usepackage{fancyhdr}
\setlength{\headheight}{15pt}
\fancyhf{} % clear the header and footers
\pagestyle{fancy}
\renewcommand{\chaptermark}[1]{\markboth{\thechapter. #1}{\thechapter. #1}}
\renewcommand{\sectionmark}[1]{\markright{\thesection. #1}} 
\renewcommand{\headrulewidth}{0pt}





% UL page number position 
\fancyfoot[C]{\emph{\thepage}} %regular pages
\fancypagestyle{plain}{\fancyhf{}\fancyfoot[C]{\emph{\thepage}}} %chapter pages




%%%%% SELECT YOUR DRAFT OPTIONS
% This adds a "DRAFT" footer to every normal page.  (The first page of each chapter is not a "normal" page.)

% IP feb 2021: option to include line numbers in PDF

% for line wrapping in code blocks
\usepackage{fancyvrb}
\usepackage{fvextra}
\DefineVerbatimEnvironment{Highlighting}{Verbatim}{breaklines=true, breakanywhere=true, commandchars=\\\{\}}

% This highlights (in blue) corrections marked with (for words) \mccorrect{blah} or (for whole
% paragraphs) \begin{mccorrection} . . . \end{mccorrection}.  This can be useful for sending a PDF of
% your corrected thesis to your examiners for review.  Turn it off, and the blue disappears.


%%%%% BIBLIOGRAPHY SETUP
% Note that your bibliography will require some tweaking depending on your department, preferred format, etc.
% If you've not used LaTeX before, I recommend just using pandoc for citations -- this is what's used unless you specific e.g. "citation_package: natbib" in index.Rmd
% If you're already a LaTeX pro and are used to natbib or something, modify as necessary.

% this allows the latex template to handle pandoc citations




% Uncomment this if you want equation numbers per section (2.3.12), instead of per chapter (2.18):
%\numberwithin{equation}{subsection}


%%%%% THESIS / TITLE PAGE INFORMATION
% Everybody needs to complete the following:




% Master's candidates who require the alternate title page (with candidate number and word count)
% must also un-comment and complete the following three lines:

% Uncomment the following line if your degree also includes exams (eg most masters):
%\renewcommand{\submittedtext}{Submitted in partial completion of the}
% Your full degree name.  (But remember that DPhils aren't "in" anything.  They're just DPhils.)


% Term and year of submission, or date if your board requires (eg most masters)



%%%%% YOUR OWN PERSONAL MACROS
% This is a good place to dump your own LaTeX macros as they come up.

% To make text superscripts shortcuts
\renewcommand{\th}{\textsuperscript{th}} % ex: I won 4\th place
\newcommand{\nd}{\textsuperscript{nd}}
\renewcommand{\st}{\textsuperscript{st}}
\newcommand{\rd}{\textsuperscript{rd}}

%%%%% THE ACTUAL DOCUMENT STARTS HERE
\begin{document}

%%%%% CHOOSE YOUR LINE SPACING HERE
% This is the official option.  Use it for your submission copy and library copy:
\setlength{\textbaselineskip}{22pt plus2pt}
% This is closer spacing (about 1.5-spaced) that you might prefer for your personal copies:
%\setlength{\textbaselineskip}{18pt plus2pt minus1pt}

% You can set the spacing here for the roman-numbered pages (acknowledgements, table of contents, etc.)
\setlength{\frontmatterbaselineskip}{17pt plus1pt minus1pt}

% UL: You can set the line and paragraph spacing here for the separate abstract page to be handed in to Examination schools
\setlength{\abstractseparatelineskip}{13pt plus1pt minus1pt}
\setlength{\abstractseparateparskip}{0pt plus 1pt}

% UL: You can set the general paragraph spacing here - I've set it to 2pt (was 0) so
% it's less claustrophobic
\setlength{\parskip}{2pt plus 1pt}

%
% Customise title page
%
\def\crest{}
\renewcommand{\university}{}
\renewcommand{\submittedtext}{}
\renewcommand{\thesistitlesize}{\fontsize{22pt}{28pt}\selectfont}
\renewcommand{\gapbeforecrest}{25mm}
\renewcommand{\gapaftercrest}{25mm
}


% Leave this line alone; it gets things started for the real document.
\setlength{\baselineskip}{\textbaselineskip}


%%%%% CHOOSE YOUR SECTION NUMBERING DEPTH HERE
% You have two choices.  First, how far down are sections numbered?  (Below that, they're named but
% don't get numbers.)  Second, what level of section appears in the table of contents?  These don't have
% to match: you can have numbered sections that don't show up in the ToC, or unnumbered sections that
% do.  Throughout, 0 = chapter; 1 = section; 2 = subsection; 3 = subsubsection, 4 = paragraph...

% The level that gets a number:
\setcounter{secnumdepth}{2}
% The level that shows up in the ToC:
\setcounter{tocdepth}{1}


%%%%% ABSTRACT SEPARATE
% This is used to create the separate, one-page abstract that you are required to hand into the Exam
% Schools.  You can comment it out to generate a PDF for printing or whatnot.

% JEM: Pages are roman numbered from here, though page numbers are invisible until ToC.  This is in
% keeping with most typesetting conventions.
\begin{romanpages}

% Title page is created here

%%%%% DEDICATION

%%%%% ACKNOWLEDGEMENTS


%%%%% ABSTRACT


%%%%% MINI TABLES
% This lays the groundwork for per-chapter, mini tables of contents.  Comment the following line
% (and remove \minitoc from the chapter files) if you don't want this.  Un-comment either of the
% next two lines if you want a per-chapter list of figures or tables.
\dominitoc % include a mini table of contents

% This aligns the bottom of the text of each page.  It generally makes things look better.
\flushbottom

% This is where the whole-document ToC appears:


% Uncomment to generate a list of tables:
%%%%% LIST OF ABBREVIATIONS
% This example includes a list of abbreviations.  Look at text/abbreviations.tex to see how that file is
% formatted.  The template can handle any kind of list though, so this might be a good place for a
% glossary, etc.

% The Roman pages, like the Roman Empire, must come to its inevitable close.
\end{romanpages}

%%%%% CHAPTERS
% Add or remove any chapters you'd like here, by file name (excluding '.tex'):
\flushbottom

% all your chapters and appendices will appear here
{word count 1335/1283}

\hypertarget{statistical-analysis}{%
\chapter{Statistical Analysis}\label{statistical-analysis}}

Table 4.1 below presents summary statistics for my sample, showing the count and relative frequency of each code per actor, ideology, region, and overall. The literature review critiqued gastropopulism studies generalising from the empirical evidence disproportionately drawn from one actor. In my sample, UK actors generated the most data (67\%), with Corbyn accounting for 63.4\% `left' data and Farage 69.9\% `right' data. This is congruent with my performance-based approach, which does not expect actors to use gastropopulism equally. Table 4.1 illustrates how conflating gastropopulism with an actor's ideology or region obscures individual-level variations and patterns. In fact, it demonstrates how gastropopulism research is unsuited to statistical generalisability. Resultantly, Table 4.1 is intended towards analytical generalisability. This chapter establishes the empirical grounding of the subsequent analytical sections.

\begin{table}

\caption{\label{tab:unnamed-chunk-1}Summary Statistics Table by Actor, Ideology, Region, and Overall}
\centering
\fontsize{14}{16}\selectfont
\begin{tabu} to \linewidth {>{}l|>{\centering}X>{\centering}X>{\centering}X>{}c|>{\centering}X>{}c|>{\centering}X>{}c|>{\centering}X}
\toprule
\multicolumn{1}{c}{ } & \multicolumn{4}{c}{\textbf{Actor}} & \multicolumn{2}{c}{\textbf{Ideology}} & \multicolumn{2}{c}{\textbf{Geography}} & \multicolumn{1}{c}{ } \\
\cmidrule(l{3pt}r{3pt}){2-5} \cmidrule(l{3pt}r{3pt}){6-7} \cmidrule(l{3pt}r{3pt}){8-9}
\textbf{Variable} & \makecell[c]{\textbf{AOC}\ \ \\N = 26} & \makecell[c]{\textbf{Corbyn}\ \ \\N = 45} & \makecell[c]{\textbf{Farage}\ \ \\N = 65} & \makecell[c]{\textbf{Trump}\ \ \\N = 27} & \makecell[c]{\textbf{Left}\ \ \\N = 71} & \makecell[c]{\textbf{Right}\ \ \\N = 92} & \makecell[c]{\textbf{UK}\ \ \\N = 110} & \makecell[c]{\textbf{USA}\ \ \\N = 53} & \makecell[c]{\textbf{Overall}\ \ \\N = 163}\\
\midrule
**Incorporation** &  &  &  &  &  &  &  &  & \\
\hspace{1em}Absent & 14\ \ \ \ (54\%) & 29\ \ \ \ (64\%) & 25\ \ \ \ (38\%) & 17\ \ \ \ (63\%) & 43\ \  (61\%) & 42\ \  (46\%) & 54\ \ \ \ (49\%) & 31\ \ \ \ (58\%) & 85\ \ (52\%)\\
\hspace{1em}Implied & 5\ \ \ \ (19\%) & 9\ \ \ \ (20\%) & 30\ \ \ \ (46\%) & 8\ \ \ \ (30\%) & 14\ \  (20\%) & 38\ \  (41\%) & 39\ \ \ \ (35\%) & 13\ \ \ \ (25\%) & 52\ \ (32\%)\\
\hspace{1em}Present & 7\ \ \ \ (27\%) & 7\ \ \ \ (16\%) & 10\ \ \ \ (15\%) & 2\ \ \ \ (7.4\%) & 14\ \  (20\%) & 12\ \  (13\%) & 17\ \ \ \ (15\%) & 9\ \ \ \ (17\%) & 26\ \ (16\%)\\
**Cultural Association** &  &  &  &  &  &  &  &  & \\
\addlinespace
\hspace{1em}Assimilated & 4\ \ \ \ (15\%) & 11\ \ \ \ (24\%) & 9\ \ \ \ (14\%) & 5\ \ \ \ (19\%) & 15\ \  (21\%) & 14\ \  (15\%) & 20\ \ \ \ (18\%) & 9\ \ \ \ (17\%) & 29\ \ (18\%)\\
\hspace{1em}Endogenous & 15\ \ \ \ (58\%) & 19\ \ \ \ (42\%) & 38\ \ \ \ (58\%) & 19\ \ \ \ (70\%) & 34\ \  (48\%) & 57\ \  (62\%) & 57\ \ \ \ (52\%) & 34\ \ \ \ (64\%) & 91\ \ (56\%)\\
\hspace{1em}Exogenous & 3\ \ \ \ (12\%) & 10\ \ \ \ (22\%) & 7\ \ \ \ (11\%) & 0\ \ \ \ (0\%) & 13\ \  (18\%) & 7\ \  (7.6\%) & 17\ \ \ \ (15\%) & 3\ \ \ \ (5.7\%) & 20\ \ (12\%)\\
\hspace{1em}N/A & 0\ \ \ \ (0\%) & 2\ \ \ \ (4.4\%) & 4\ \ \ \ (6.2\%) & 0\ \ \ \ (0\%) & 2\ \  (2.8\%) & 4\ \  (4.3\%) & 6\ \ \ \ (5.5\%) & 0\ \ \ \ (0\%) & 6\ \ (3.7\%)\\
\hspace{1em}Regional Endogenous & 4\ \ \ \ (15\%) & 3\ \ \ \ (6.7\%) & 7\ \ \ \ (11\%) & 3\ \ \ \ (11\%) & 7\ \  (9.9\%) & 10\ \  (11\%) & 10\ \ \ \ (9.1\%) & 7\ \ \ \ (13\%) & 17\ \ (10\%)\\
\addlinespace
**Nourishment** &  &  &  &  &  &  &  &  & \\
\hspace{1em}Healthy & 1\ \ \ \ (3.8\%) & 7\ \ \ \ (16\%) & 8\ \ \ \ (12\%) & 1\ \ \ \ (3.7\%) & 8\ \  (11\%) & 9\ \  (9.8\%) & 15\ \ \ \ (14\%) & 2\ \ \ \ (3.8\%) & 17\ \ (10\%)\\
\hspace{1em}Moderate & 6\ \ \ \ (23\%) & 19\ \ \ \ (42\%) & 4\ \ \ \ (6.2\%) & 3\ \ \ \ (11\%) & 25\ \  (35\%) & 7\ \  (7.6\%) & 23\ \ \ \ (21\%) & 9\ \ \ \ (17\%) & 32\ \ (20\%)\\
\hspace{1em}Unhealthy & 19\ \ \ \ (73\%) & 19\ \ \ \ (42\%) & 53\ \ \ \ (82\%) & 23\ \ \ \ (85\%) & 38\ \  (54\%) & 76\ \  (83\%) & 72\ \ \ \ (65\%) & 42\ \ \ \ (79\%) & 114\ \ (70\%)\\
**Taste** &  &  &  &  &  &  &  &  & \\
\addlinespace
\hspace{1em}Alcohol & 5\ \ \ \ (19\%) & 3\ \ \ \ (6.7\%) & 35\ \ \ \ (54\%) & 0\ \ \ \ (0\%) & 8\ \  (11\%) & 35\ \  (38\%) & 38\ \ \ \ (35\%) & 5\ \ \ \ (9.4\%) & 43\ \ (26\%)\\
\hspace{1em}Fresh & 3\ \ \ \ (12\%) & 5\ \ \ \ (11\%) & 8\ \ \ \ (12\%) & 2\ \ \ \ (7.4\%) & 8\ \  (11\%) & 10\ \  (11\%) & 13\ \ \ \ (12\%) & 5\ \ \ \ (9.4\%) & 18\ \ (11\%)\\
\hspace{1em}Hot drink & 2\ \ \ \ (7.7\%) & 7\ \ \ \ (16\%) & 3\ \ \ \ (4.6\%) & 0\ \ \ \ (0\%) & 9\ \  (13\%) & 3\ \  (3.3\%) & 10\ \ \ \ (9.1\%) & 2\ \ \ \ (3.8\%) & 12\ \ (7.4\%)\\
\hspace{1em}Other & 3\ \ \ \ (12\%) & 4\ \ \ \ (8.9\%) & 0\ \ \ \ (0\%) & 0\ \ \ \ (0\%) & 7\ \  (9.9\%) & 0\ \  (0\%) & 4\ \ \ \ (3.6\%) & 3\ \ \ \ (5.7\%) & 7\ \ (4.3\%)\\
\hspace{1em}Salty/fatty & 7\ \ \ \ (27\%) & 20\ \ \ \ (44\%) & 11\ \ \ \ (17\%) & 16\ \ \ \ (59\%) & 27\ \  (38\%) & 27\ \  (29\%) & 31\ \ \ \ (28\%) & 23\ \ \ \ (43\%) & 54\ \ (33\%)\\
\addlinespace
\hspace{1em}Sweet & 6\ \ \ \ (23\%) & 6\ \ \ \ (13\%) & 8\ \ \ \ (12\%) & 9\ \ \ \ (33\%) & 12\ \  (17\%) & 17\ \  (18\%) & 14\ \ \ \ (13\%) & 15\ \ \ \ (28\%) & 29\ \ (18\%)\\
**Worker Performance** &  &  &  &  &  &  &  &  & \\
\hspace{1em}Absent & 22\ \ \ \ (85\%) & 36\ \ \ \ (80\%) & 61\ \ \ \ (94\%) & 27\ \ \ \ (100\%) & 58\ \  (82\%) & 88\ \  (96\%) & 97\ \ \ \ (88\%) & 49\ \ \ \ (92\%) & 146\ \ (90\%)\\
\hspace{1em}Present & 4\ \ \ \ (15\%) & 9\ \ \ \ (20\%) & 4\ \ \ \ (6.2\%) & 0\ \ \ \ (0\%) & 13\ \  (18\%) & 4\ \  (4.3\%) & 13\ \ \ \ (12\%) & 4\ \ \ \ (7.5\%) & 17\ \ (10\%)\\
**Part of 'the people'** &  &  &  &  &  &  &  &  & \\
\addlinespace
\hspace{1em}Absent & 0\ \ \ \ (0\%) & 1\ \ \ \ (2.2\%) & 0\ \ \ \ (0\%) & 0\ \ \ \ (0\%) & 1\ \  (1.4\%) & 0\ \  (0\%) & 1\ \ \ \ (0.9\%) & 0\ \ \ \ (0\%) & 1\ \ (0.6\%)\\
\hspace{1em}Present & 26\ \ \ \ (100\%) & 44\ \ \ \ (98\%) & 65\ \ \ \ (100\%) & 27\ \ \ \ (100\%) & 70\ \  (99\%) & 92\ \  (100\%) & 109\ \ \ \ (99\%) & 53\ \ \ \ (100\%) & 162\ \ (99\%)\\
**Bad Manners** &  &  &  &  &  &  &  &  & \\
\hspace{1em}Absent & 2\ \ \ \ (7.7\%) & 11\ \ \ \ (24\%) & 2\ \ \ \ (3.1\%) & 2\ \ \ \ (7.4\%) & 13\ \  (18\%) & 4\ \  (4.3\%) & 13\ \ \ \ (12\%) & 4\ \ \ \ (7.5\%) & 17\ \ (10\%)\\
\hspace{1em}Present & 24\ \ \ \ (92\%) & 34\ \ \ \ (76\%) & 63\ \ \ \ (97\%) & 25\ \ \ \ (93\%) & 58\ \  (82\%) & 88\ \  (96\%) & 97\ \ \ \ (88\%) & 49\ \ \ \ (92\%) & 146\ \ (90\%)\\
\addlinespace
**Immediacy** &  &  &  &  &  &  &  &  & \\
\hspace{1em}Absent & 0\ \ \ \ (0\%) & 2\ \ \ \ (4.4\%) & 0\ \ \ \ (0\%) & 0\ \ \ \ (0\%) & 2\ \  (2.8\%) & 0\ \  (0\%) & 2\ \ \ \ (1.8\%) & 0\ \ \ \ (0\%) & 2\ \ (1.2\%)\\
\hspace{1em}Present & 26\ \ \ \ (100\%) & 43\ \ \ \ (96\%) & 65\ \ \ \ (100\%) & 27\ \ \ \ (100\%) & 69\ \  (97\%) & 92\ \  (100\%) & 108\ \ \ \ (98\%) & 53\ \ \ \ (100\%) & 161\ \ (99\%)\\
**Individual Trend** &  &  &  &  &  &  &  &  & \\
\hspace{1em}Absent & 3\ \ \ \ (12\%) & 19\ \ \ \ (46\%) & 22\ \ \ \ (34\%) & 8\ \ \ \ (30\%) & 22\ \  (33\%) & 30\ \  (33\%) & 41\ \ \ \ (39\%) & 11\ \ \ \ (21\%) & 52\ \ (33\%)\\
\addlinespace
\hspace{1em}Implied & 7\ \ \ \ (27\%) & 13\ \ \ \ (32\%) & 16\ \ \ \ (25\%) & 10\ \ \ \ (37\%) & 20\ \  (30\%) & 26\ \  (28\%) & 29\ \ \ \ (27\%) & 17\ \ \ \ (32\%) & 46\ \ (29\%)\\
\hspace{1em}Present & 16\ \ \ \ (62\%) & 9\ \ \ \ (22\%) & 27\ \ \ \ (42\%) & 9\ \ \ \ (33\%) & 25\ \  (37\%) & 36\ \  (39\%) & 36\ \ \ \ (34\%) & 25\ \ \ \ (47\%) & 61\ \ (38\%)\\
\bottomrule
\end{tabu}
\end{table}

\hypertarget{incorporation}{%
\section*{Incorporation}\label{incorporation}}
\addcontentsline{toc}{section}{Incorporation}

While Diehl (2017, p12) and Bourdieu (1984, p190) emphasise incorporation, or eating, as the most powerful way to signify belonging to the people, Table 4.1 shows that this was present in only 26 (16\%) items overall. `Implied' incorporation, with the actor framed as ``about'' to eat/drink was twice as common, found in 52 (32\%) items. Just over half of my sample lacked actual or implied incorporation.

Interestingly, breaking down the \texttt{Data\ Format} variable by the \texttt{Incorporation} variable indicates the affordances of audiovisual content for incorporation {{[}table in Appendix B{]}}. This shows that 23 (88\%) of the 26 `present' \texttt{incorporation} items were audiovisual, and 50 (96\%) of the 52 `implied' \texttt{incorporation} items were visual. Still images of a political actor incorporating food/drink present a high risk of looking overly posed (breaking the illusion of spontaneity) or overly unappealing (inviting ridicule). With still images, implied incorporation may appear more natural/candid, thus legitimising the performed authenticity and spontaneity. Biting and chewing are dynamic, violent motions (Stano, 2015, p658) best suited to the video format, which captures the process of incorporation rather than just a frame. With videos, implied incorporation may pierce performed authenticity and spontaneity, as it highlights that the actor is not actually consuming the food/drink. This speaks to the distinct affordances of visual and audiovisual formats for impression management.
I return to this idea in my final analytical section, §7: Performative Eating.

\hypertarget{cultural-association}{%
\section*{Cultural Association}\label{cultural-association}}
\addcontentsline{toc}{section}{Cultural Association}

Examining the \texttt{Cultural\ Association} variable provides initial support for my theory's emphasis on the explanatory power of individual-level analysis rather than extant gastropopulism theory's focus on ideology and Italy.
Food semiotics are highly culturally situated (Kress, 2011, p45). Accordingly, the categories for this variable are situated, meaning they relate to the cultural identities of the actor and their region. This adapts the food origin distinctions presented as national/local by Starita (2022, Appendix) and as endogenous/exogenous by Stano (2015, p3) {put in codebook}. For example, the Scottish soft drink Irn Bru is `regional endogenous' for Farage {[}P3{]} and Corbyn {[}P2{]} but `exogenous' for AOC {[}P1; V49{]}. Farage's appearances at the \emph{British Curry} Awards {[}P109; V10; V11; V13{]} and Corbyn's at the \emph{British Kebab} Awards {[}V12; V14{]} are coded as `assimilated'. Conversely, items showing Corbyn dishing up curries in Sikh Gurdwaras {[}P14; P34; P39; V30{]} are coded as `exogenous', as Corbyn is not Sikh.

Unexpectedly, Table 4.1 shows that AOC and Farage have the most similar \texttt{cultural\ association} distributions, with almost identical relative frequencies of their uses of endogenous, assimilated, and exogenous foods. This is notable because their politics surrounding cultural/national identity are diametrically opposed. Prior gastropopulism literature emphasises an exclusionary/inclusionary dichotomy, asserting that right-wing populists primarily use food to convey their exclusionary nationalism, with no reference to its potential for inclusion. In §6.1: Nationalism, I explore how the strategic ambiguity utilised in gastropopulist performances by Trump and Farage thwart analytical attempts to code gastropopulism using an exclusionary/inclusionary dichotomy. The individual variation and strategic ambiguity within my sample's \texttt{cultural\ association} variable and analysis thereof demonstrates the theoretical and empirical limitations of analysing gastropopulism through an ideological lens. My performance-based theory does not deny that ideology and region contribute to how an actor manifests their gastropopulism. However, I assert gastropopulism as a flexible and open way for individuals to construct identity, rather than a restrictive representation of ideology that exists independently of the actor.

\hypertarget{performances-of-class-belonging}{%
\section*{Performances of Class Belonging}\label{performances-of-class-belonging}}
\addcontentsline{toc}{section}{Performances of Class Belonging}

The \texttt{Nourishment} and \texttt{Taste} variables attempt to capture Bourdieu's (1984, p468) elite/mass food-based class division (e.g., healthy/unhealthy; light/heavy; quality/quantity). Table 4.1 shows that AOC, Farage, and Trump lean heavily towards unhealthy foods. Corbyn's profile is more balanced, but favours the moderate-unhealthy and salty/fatty. The \texttt{taste} variable's \texttt{fresh} category includes home-cooked meals. This is where the absence of AOC's disappearing gastropopulist content (e.g., cooking livestreams) is most keenly felt, limiting its analytical utility. The consistency of Trump's performed tastes preferences for salty/fatty {[}n=16 (59\%){]} and sweet {[}n=9 (33\%){]} foods is not a coincidence; García-Santamaría (2020, p143) views this as a way to reveal a relatable and \emph{seemingly} apolitical weakness. Farage's extensive use of alcohol in his gastropopulism {[}n=35 (54\%){]} is a form of mimetic identification with British binge drinking culture (Jayne \emph{et al}, 2008, p88). Jayne \emph{et al} (2008, p88) caution against accepting entrenched ``classed'' perceptions of ``respectability'' that oppose British binge drinking with ``\,`continental' or `sensible'\,'' European drinking culture. However, through Goffman's (1959, p20) lens, this is ``dramatic realisation'' of precisely the identity Farage wishes to convey through his alcohol-centred gastropopulist performances. In addition, the \texttt{Worker\ Performance} variable sets the stage for §6.2: Class, examining how the actors embody their class belonging through food hospitality worker role performances. No such observations of Trump were found, so his discussion explores how he embodies class belonging through his menu choices.

\hypertarget{gastropopulism-and-individual-trends}{%
\section*{Gastropopulism and Individual Trends}\label{gastropopulism-and-individual-trends}}
\addcontentsline{toc}{section}{Gastropopulism and Individual Trends}

The variables capturing gastropopulism's individual features show that the actors were highly consistent at integrating all three into their performances. Corbyn, however, omitted \texttt{bad\ manners} in 24\% his performances. This motivates looking at his full dataset, and providing a multimodal analysis of how he -- and the others -- actually perform their gastropopulism \emph{multimodally}. To this end, the \texttt{Individual\ Trend} variable sets the stage for the next analytical chapter: Individual Timelines and Trends.

\textbf{Trump} {[}n=9 (33\%){]} \textbf{`put his money where his mouth is'}, meaning instances wherein he (claimed to) use his personal wealth to buy fast food for his people, with a secondary code for performances of his personal consumption of popular, cheap foods {[}n=10 (37\%){]}. \textbf{AOC} {[}n=16 (62\%){]} \textbf{physically engages with her constituency} through gastropopulist performances, with a secondary code for performances implied to be in her NY/DC homes {[}n=7 (27\%){]}. \textbf{Corbyn} {[}n=9 (22\%){]} uses his Labour campaign slogan \textbf{``for the many, not the few''} in his gastropopulist performances, particularly through puns and emojis, with a secondary code for performances embodying the slogan, e.g., community work {[}n=13 (32\%){]}. Finally, I explore how \textbf{Farage} {[}n=27 (42\%){]} explicitly leverages \textbf{food and drink as the cause and solution of problems}, blending the personal and the political, with a secondary code for doing so implicitly {[}n=16 (25\%){]}.

\hypertarget{discussion-of-statistical-analysis}{%
\section*{Discussion of Statistical Analysis}\label{discussion-of-statistical-analysis}}
\addcontentsline{toc}{section}{Discussion of Statistical Analysis}

This statistical analysis chapter has assessed how political actors construct gastropopulist performances. Moreover, this has provided a robust comparison of individual-, ideological-, and regional-level explanations. This gives an unprecedented depth and breadth of analysis of empirical evidence of gastropopulism. Crucially, this facilitates analytic generalisability for existing theory as well as establishing a flexible and transparent framework for future studies of (gastro)populism. In addition, this has demonstrated how my research design is informed by the data as well as literature, in order to conduct a cohesive and comprehensive test of my theory. This chapter has contextualised the empirical grounding for the subsequent analyses, which will closely scrutinise the multimodality of the performances, and the role of constructed public identity. This will enable a considered answer to my research question, \emph{how do political actors use multimodal gastropopulist performances to construct and legitimise their public identities?}.

%%%%% REFERENCES


\end{document}
